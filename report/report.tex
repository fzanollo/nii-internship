\documentclass[a4paper]{report}
 
\usepackage{graphicx}
\usepackage{url}
\usepackage{subcaption}
\usepackage[section]{placeins} % Ensure floats do not go into the next section
\usepackage{rotating} % For rotating figures, tables, etc. Including their captions

\begin{document}
	\title{Overview of anime voice actor's social network and popularity.}
	\author{Florencia Zanollo. Hideaki Takeda.}
	\maketitle
	\tableofcontents

\begin{abstract}
	Although Social Network of actors is a relatively common object of investigation that has been addressed many times, we can say that seiyuu (anime voice actors) come from a very different industry with a distinct way to relate to each other. \\
	In this research we use Wikidata and MyAnimeList to collect seiyuu information and build a Social Network. Topics explored:
	\begin{itemize}
	\item Structure and characteristics of seiyuu Social Network.
	\item Understanding what properties have a main role describing and predicting popularity of seiyuu.
	\item Compare prediction performances between different machine learning algorithms (and different models).
	\end{itemize}
\end{abstract}

\chapter*{Introduction}
\addcontentsline{toc}{chapter}{Introduction}
	TODO

% Main Part
\section{Anime/Seiyuu Dataset}

\PARstart{W}{ikidata}\footnote{http://wikidata.org/} is a collaboratively edited knowledge base intended to provide a common source of data which can be used by Wikimedia projects such as Wikipedia. The information is stored in RDF format, and can be retrieved in multiple ways, one of them being through a SPARQL endpoint.

Using Wikidata's SPARQL endpoint we retrieved a list of seiyuu. This list contains all persons that have seiyuu as occupation, a total of 6472 entities were obtained\footnote{There's actually 7030 seiyuu in Wikidata but only 6472 of them have an English label (name)}. Gender, birthday and birthplace information was also fetched (last two were not used in the end because it was lacking in the majority of entities).

Since Wikidata information about seiyuu's works is really incomplete, MyAnimeList (MAL)\footnote{https://myanimelist.net/} was used to retrieve voice acting roles and anime information. MAL is a social networking and social cataloging application website with a large database on anime and manga that started in April 6, 2006. Users can make a list of currently watching, watched and/or favorite anime; also score, review, comment and recommend similar ones. They can also put comments and favorite people working on the industry (voice actors, directors, editors, etc).

Since only 59 of Wikidata's seiyuu entities had MyAnimeList ID (MALID) property; a matching between Wikidata and MyAnimeList was done using seiyuu's complete name to retrieve the ID for those who didn’t had. Successfully restoring 3033 MALIDs, giving a total of 3092 seiyuus with that property; 2956 of them have at least one work according to MAL so we are using this subset for our experiments.

Using Jikan API\footnote{https://jikan.docs.apiary.io/\#} and MALID, seiyuu data, voice acting roles and more information about each anime was retrieved. 

An issue to take into account is whether we unify all anime adaptations of the same intellectual property as one or take a single adaptation as a independent work. We choose the later because a seiyuu could work at one adaptation only, which has its own producer, score, popularity, among other information; it would be incorrect to say a seiyuu worked in a popular work when actually that adaptation didn’t have enough fame.

Information that was ultimately used:
\begin{itemize}
	\item For Seiyuu:
	\begin{itemize}
		\item Name
		\item Debut (this was obtained from oldest work aired date)
		\item Gender
		\item Popularity (member\_favorites information of MAL)
		\item Work (anime roles)
	\end{itemize}
	\item For Works (Anime):
	\begin{itemize}
		\item Year that began airing
		\item Favorites
		\item Score
		\item Popularity
		\item Genres
	\end{itemize}
\end{itemize}

It's important to notice that data such as popularity and scores are retrieved from MAL, which is user review based only, so it may differ with actual awards winning or professional reviewing of works.

Also, this dataset is biased in favor of more recent anime and seiyuu, since it accounts for more complete data and with better quality. Oldest anime in this dataset is from 1960 having no record about previous ones. Majority of seiyuu's debut are from 1988 which leads us to think information from thereon is more complete. 

The data was stored using Virtuoso server to create a local SPARQL endpoint, mongodb was also used as an intermediate storage (before formatting data as RDF).












\section{Seiyuu Social Network}
TODO add little explanation about social network

\subsection{Node and edge definition}
Our social network consists of voice actors (seiyuu) as nodes and co-workership between them as edges. It's important to notice that this social network is time dependant since each seiyuu has a debut year and each anime has an aired time; giving us freedom to choose different time frames to observe it.

Aside from being time dependant there exists different possible definitions of relationship or co-workership between seiyuu. One could say two actors know each other if they have worked in at least one job together, or maybe it requires more than one. There’s also a time frame to define, relationship could take into account all works of both of them or only of a certain time frame.

After observing graphs built with different interpretations of relationship, the criteria for connecting two nodes became: at least 10 works in common, during the time frame between the first debut registered (1960) and the year of observation.
The reason behind this decision is that requiring more jobs in common means less amount of edges; this leaves a more understandable graph without changing its structure.

There's also other interesting definitions of relationship, for example we can use only common works from the last x years. This options weren't explored; having into account our limited time we opted to decide on one and put more effort in analyzing the data and social network at hand.

\subsection{Construction}
As a first approach Gephi was used to build the network. Since the graph was big enough to bring performance problems and we needed to build the edges dynamically (which couldn't be done in Gephi) NetworkX was used instead.

NetworkX was chosen because it's an easy yet powerful Python library, it doesn't get along with massive graphs but ours was not big enough to present a problem. One can also export the graph and open it on Gephi, for a more visual analysis.

We needed to build the edges dynamically because they depend on the time frame we are looking at. For example if two actors worked together in 9 jobs between 1960 and 1970 we shouldn't see an edge between them; but if they worked together again in 1971 then looking at 1960-1971 they should be connected.

\subsection{Analysis}
Is easy to tell at first glance that this social network is really interconnected. With only 2956 nodes it has 395887 edges when only one work in common is required and 13629 edges when asking for 10 or more. It shows a thightly interconnected cluster surrounded by poorly or not connected nodes. This cluster represents 




\chapter{Analysis and prediction of seiyuu popularity}

Popularity is an abstract criterion that must be defined as a numerical metric in order to be used for analysis and prediction. Since we are using MAL database and it has a social component, seems logic to use member\_favorites as a representation of popularity. We can also get popularity and score of anime from opinions of the same set of users.\\

In terms of distribution \textit{popularity} is highly unequal \---as we can observe in Fig.~\ref{fig:popularityDistribution}\--- having a lot of seiyuu which are no member favorites and only a few who are favorite of more than 10000 members. It's good to keep in mind that users can favorite multiple seiyuu.

\begin{figure}[!h]
	\begin{center}
	\includegraphics[width=\columnwidth]{graphics/popularityDistribution.png}
	\caption{Amount of seiyuu with that popularity, divided into groups for better visualization.}
	\label{fig:popularityDistribution}
	\end{center}
\end{figure}

\newpage

Some metrics about popularity:
\begin{itemize}
	\item Mean:    289.55
	\item Median:    2.0
	\item Max:    55018
	\item Min:    0 
	\item 1037 values equal to zero
	\item Only 120 values bigger than 1000
\end{itemize}

\begin{table}[!h]
	\begin{center}
	\caption{Top 10 popular seiyuu}
	\label{tab:top10Popularity}
	\begin{tabular}{|l|c|}
		\hline
		Name & Popularity \\ 
		\hline
		Kana Hanazawa & 56637 \\ 
		\hline
		Hiroshi Kamiya & 49685 \\ 
		\hline
		Mamoru Miyano & 43942 \\ 
		\hline
		Rie Kugimiya & 31668 \\ 
		\hline
		Jun Fukuyama & 26811 \\ 
		\hline
		Miyuki Sawashiro & 26501 \\ 
		\hline
		Tomokazu Sugita & 24449 \\ 
		\hline
		Daisuke Ono & 24080 \\ 
		\hline
		Saori Hayami & 18322 \\ 
		\hline
		Aya Hirano & 18094 \\ 
		\hline
	\end{tabular}
	\end{center}
\end{table}

\newpage
\section{Correlation with only one feature}
Our first approach to explaining popularity was using Pearson correlation.

\begin{figure}[!h]
	\begin{flushleft}
	\makebox[\textwidth][c]{\includegraphics[width=1.7\columnwidth]{graphics/10Works_correlation_Pearson_allWorks.png}}%
	\caption{Pearson correlation between popularity and attribute of nodes (using all works).}
	\label{fig:pearsonCorrAllWorks}
	\end{flushleft}
\end{figure}

As shown on Fig.~\ref{fig:pearsonCorrAllWorks} a fairly big correlation can be seen between popularity and amount of works. This attribute doesn't have the biggest correlation with popularity but "number of main roles" was added to the end of this investigation since we didn't had the data for doing so before. 
We are showing only average of values for work's attributes (ex. favorites) mostly for better visualization but for predictions we use sum, mean, median and maximum.

Since our dataset is biased in favor of more modern anime we thought of correlate with more recent works only. But, how recent? Last 5, 10 or 20 years? Thus correlation between popularity and works from different data frames was analyzed, Fig.~\ref{fig:correlationPopRecentWorks}.

\begin{figure}[!h]
	\begin{center}
	\includegraphics[width=\columnwidth]{graphics/correlationPopRecentWorks.png}
	\caption{Last \textit{X} years means works from 2018-\textit{X} to present.}
	\label{fig:correlationPopRecentWorks}
	\end{center}
\end{figure}

The best result was given by recent works from last 9 years. Therefore, this definition of recent works was used from there on.

Fig.~\ref{fig:pearsonCorrRecentWorks} shows the result of running Pearson again but using information from last 9 years of works only. It's important to clarify that we didn't build the network using only last 9 years, so betweenness centrality and degree are exactly the same as before. We left "amount of works" attribute for easy comparision against "amount of recent works".

\begin{figure}[!h]
	\begin{flushleft}
	\makebox[\textwidth][c]{\includegraphics[width=1.7\columnwidth]{graphics/10Works_correlation_Pearson_recentWorks.png}}%
	\caption{Pearson correlation between popularity and attribute of nodes (using recent works only).}
	\label{fig:pearsonCorrRecentWorks}
	\end{flushleft}
\end{figure}

\FloatBarrier
\subsection{Why last 9 years of works has more correlation?}
Graphics of some characteristics of works divided by years were made, trying to shed some light over why works from last 9 years were more "important".

\begin{sidewaysfigure}
	\centering
	\begin{subfigure}{.55\columnwidth}
		\centering
		\includegraphics[width=\columnwidth]{graphics/avgFavorites.png}
		\caption{Average amount of favorites per year.}
		\label{fig:avgFavorites}
	\end{subfigure}%
	\begin{subfigure}{.55\columnwidth}
		\centering
		\includegraphics[width=\columnwidth]{graphics/avgScores.png}
		\caption{Average score per year.}
		\label{fig:avgScores}
	\end{subfigure}
	\begin{subfigure}{.55\columnwidth}
		\centering
		\includegraphics[width=\columnwidth]{graphics/avgPopularities.png}
		\caption{Average popularity per year.}
		\label{fig:avgPopularities}
	\end{subfigure}%
	\begin{subfigure}{.55\columnwidth}
		\centering
		\includegraphics[width=\columnwidth]{graphics/avgMembers.png}
		\caption{Average amount of members per year.}
		\label{fig:avgMembers}
	\end{subfigure}
	\caption{Averages of some atributes of anime divide by year.}
	\label{fig:averages}
\end{sidewaysfigure}

Fig.~\ref{fig:avgFavorites} and ~\ref{fig:avgScores} shows an improvement in average of scores and favorites. 
Biggest peak of favorites is on 2005, this may have to do with the start of MAL (2006), see Fig.~\ref{fig:avgFavorites}. We suppose as users started to use MAL they favorite anime they liked from that year and only some of the old ones; from then on they used the website often and favorite new works as they began airing.

Year 2018 was left out of Fig.~\ref{fig:avgScores} because, since 2018 is not finished yet, the average of scores was unusually small.

As Fig.~\ref{fig:avgPopularities} shows, popularity goes down in last years. But we need to consider that this metrics are from MAL and it could mean, for example, the parameter is in disuse; instead of representing how people feel about new anime.\\

We suppose the attribute "members" of an anime is taken from how many users have it on any of their lists. If that is correct Fig.~\ref{fig:avgMembers} tells us feature of adding anime in watched / watching / plan to watch lists is really used. As for our experience on the web and social media we can confirm this is the most used feature of MAL. So this makes it one of the best metrics to measure "public" as another sense of popularity of anime.\\

As we can see on Fig.~\ref{fig:amountOfWorksPerYear} anime industry is growing bigger each year, of course this is biased by the fact MAL will sure have every adaptation of last year but maybe not for anime from 1980.

\begin{figure}[!h]
	\begin{center}
	\includegraphics[width=\columnwidth]{graphics/worksPerYear_1960-2018.png}
	\caption{Amount of works divided by years which were aired for the first time.}
	\label{fig:amountOfWorksPerYear}
	\end{center}
\end{figure}

The majority of works are from 1990 to 2018 and half of them are distributed over the last 14 years (2014 to 2018) but as far as we can tell there isn’t anything particular over the last 9 years nor on year 2009. Judging by amount of favorites per year it appears that users have been more active in recent years so this could be one of the reasons.

\section{Correlation with multiple features}
For this section Scikit-learn, a free software machine learning Python library, was used. The node attributes were divided into categories, leaving four distinct types:

\begin{itemize}
	\item Personal data:
	\begin{itemize}
		\item Debut
		\item Gender
		\item Activity years (2018-debut)
	\end{itemize}
	\item Works data:
	\begin{itemize}
		\item Amount
		\item Top 5 genre
		\item Favorites
		\item Score
		\item Popularity
		\item Members
	\end{itemize}
	\item Recent works data:
	\begin{itemize}
		\item Same as works but for only last 9 years
	\end{itemize}	
	\item Graph data:
	\begin{itemize}
		\item Degree
		\item Betweenness centrality
		\item Closeness
	\end{itemize}
\end{itemize}

Fitting and prediction experiments were run for each category, each combination of 2, 3 and all of them together; using 80\% of seiyuu as train data and the rest as test. This was done for all following models:
\begin{itemize}
	\item DecisionTreeRegressor
	\item DecisionTreeClassifier
	\item LinearRegression
	\item KNeighborsClassifier
	\item LinearDiscriminantAnalysis
	\item GaussianNB
	\item SVM
\end{itemize}

We have to take into account popularity variance when trying to predict it since usual error metrics don't distinguish between small or big values. If we predict exactly a seiyuu that has \textgreater50000 popularity but we make "small" mistakes predicting seiyuu with \textless100 popularity then, is it a good prediction? and having into account more than $\frac{3}{4}$ of them have \textless100 popularity?

To compare prediction performance mean and median absolute error were used. Unfortunately since popularity variance is really high we observed good results in terms of absolute error but particular predictions were aloof. This is why we use r2\_score\footnote{\url{http://scikit-learn.org/stable/modules/model_evaluation.html#r2-score-the-coefficient-of-determination}} for accuracy comparation. 

\textbf{TODO ADD SOME OF THE GRAPHICS ABOUT FEATURE IMPORTANCE FOR DTC (ONLY BEST ONES) AND EXPLAIN (FOR EACH SUBSECTION)}

\FloatBarrier
\subsection{Only one category}
\begin{table}[!hbt]
	\begin{center}
	\caption{Only one category R2 score results}
	\label{tab:oneCategory}
	\begin{tabular}{lrrrr}
\toprule
{} &  Personal &  Graph &  Work &  RecentWorks \\
\midrule
DecisionTreeClassifier     &     -0.02 &  -3.26 & -2.12 &        -0.59 \\
DecisionTreeRegressor      &     -0.03 &  -1.05 & -0.96 &         0.35 \\
GaussianNB                 &     -0.47 &  -0.57 &  0.01 &         0.03 \\
KNeighborsClassifier       &     -0.02 &   0.04 &  0.07 &         0.10 \\
LinearDiscriminantAnalysis &     -0.02 &  -4.34 &  0.31 &         0.49 \\
LinearRegression           &     -0.00 &   0.13 &  0.31 &         0.48 \\
SVM                        &     -0.02 &  -0.23 & -0.02 &         0.02 \\
\bottomrule
\end{tabular}

	\end{center}
\end{table}

\FloatBarrier
\subsection{Groups of two categories}
\begin{table}[!hbt]
	\begin{center}
	\caption{Two categories R2 score results (R: recent works, P: personal, G: graph, W: work)}
	\label{tab:twoCategories}
	\begin{tabular}{lrrrrrr}
\toprule
{} &   P+G &   P+W &   P+R &   G+W &   G+R &   W+R \\
\midrule
DecisionTreeClassifier     & -1.52 &  0.24 & -1.10 &  0.20 &  0.50 & -0.87 \\
DecisionTreeRegressor      & -0.71 &  0.19 &  0.33 & -0.68 &  0.38 &  0.28 \\
GaussianNB                 & -0.01 &  0.02 &  0.04 &  0.00 &  0.07 &  0.02 \\
KNeighborsClassifier       &  0.05 &  0.09 &  0.30 &  0.24 &  0.36 &  0.37 \\
LinearDiscriminantAnalysis & -0.01 &  0.35 &  0.53 &  0.20 &  0.54 &  0.48 \\
LinearRegression           &  0.26 &  0.51 &  0.53 &  0.50 &  0.64 &  0.24 \\
SVM                        & -0.02 & -0.00 &  0.01 & -0.02 &  0.02 & -0.03 \\
\bottomrule
\end{tabular}

	\end{center}
\end{table}

\FloatBarrier
\subsection{Groups of three categories}
\begin{table}[!hbt]
	\begin{center}
	\caption{Three categories R2 score results (R: recent works, P: personal, G: graph, W: work)}
	\label{tab:threeCategories}
	\begin{tabular}{lrrrr}
\toprule
{} &  P+G+W &  P+G+R &  P+W+R &  G+W+R \\
\midrule
DecisionTreeClassifier     &  -0.90 &  -8.46 &  -0.72 &  -0.73 \\
DecisionTreeRegressor      &  -1.75 &  -2.12 &  -1.55 &   0.20 \\
GaussianNB                 &   0.01 &   0.05 &   0.04 &   0.02 \\
KNeighborsClassifier       &   0.06 &   0.64 &  -0.04 &   0.18 \\
LinearDiscriminantAnalysis &  -0.16 &  -0.19 &   0.54 &   0.51 \\
LinearRegression           &  -0.96 &  -1.63 &   0.09 &   0.38 \\
SVM                        &  -0.03 &  -0.04 &  -0.03 &  -0.02 \\
\bottomrule
\end{tabular}

	\end{center}
\end{table}

\FloatBarrier
\subsection{All categories}
\begin{table}[!hbt]
	\begin{center}
	\caption{Only one category R2 score results}
	\label{tab:allCategories}
	\begin{tabular}{lr}
\toprule
{} &  AllFeatures \\
\midrule
DecisionTreeClassifier     &        -0.19 \\
DecisionTreeRegressor      &         0.49 \\
GaussianNB                 &         0.01 \\
KNeighborsClassifier       &         0.13 \\
LinearDiscriminantAnalysis &         0.53 \\
LinearRegression           &         0.55 \\
SVM                        &         0.01 \\
\bottomrule
\end{tabular}

	\end{center}
\end{table}

\section{Seiyuu classification}
\textbf{ADD THE NEW CATEGORIZATION OF SEIYUU AND PREDICTION RESULTS (lo tengo que hacer primero)}

\section{Conclusion}


\chapter*{Conclusion}
\addcontentsline{toc}{chapter}{Conclusion}
	

\begin{thebibliography}{5}
	%Each item starts with a \bibitem{reference} command and the details thereafter.

\end{thebibliography}

\end{document}