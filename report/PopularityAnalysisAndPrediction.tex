\chapter{Analysis and prediction of seiyuu popularity}

Popularity is an abstract criterion that must be defined as a numerical metric in order to be used for analysis and prediction. Since we are using MAL database for seiyuu and anime information and it has a social component; seems logic to use member\_favorites as a representation of popularity. We can also get popularity and score for works from opinions of the same set of users.

In terms of distribution \textit{popularity} is highly unequal \---as we can observe in Fig.~\ref{fig:popularityDistribution}\--- having a lot of seiyuu which are no member favourites and only a few who are favorite of more than 10000 members. It's good to keep in mind that users can favorite multiple seiyuu.

\begin{figure}[!hbt]
	\begin{center}
	\includegraphics[width=\columnwidth]{graphics/popularityDistribution.png}
	\caption{Amount of seiyuu with that popularity, divided into groups for better visualization.}
	\label{fig:popularityDistribution}
	\end{center}
\end{figure}

EXPLICAR LO QUE SIGNIFICA QUE SEAN TAN DIFERENTES EN VALORES, MOSTRAR LOS DATOS DEL ITEMIZE
This is something to take into account when trying to predict popularity of actors or explain it using other features. TODO ADD WHY

\begin{itemize}
	\item Mean:    289.55
	\item Median:    2.0
	\item Max:    55018
	\item Min:    0 (1037 values equal to zero)
	\item Only 120 values bigger than 1000
\end{itemize}

\section{Correlation with only one feature}
As shown in Fig.~\ref{fig:pearsonCorr} our first approach to explaining popularity was using Pearson correlation. 
TODO PONER UN GRAFO DE PEARSON CON CASI TODOS LOS PARAMETROS POSIBLES

\begin{figure}[!hbt]
	\begin{center}
	\includegraphics[width=\columnwidth]{graphics/10Works_correlation_Pearson.png}
	\caption{Pearson correlation between popularity and some features of nodes.}
	\label{fig:pearsonCorr}
	\end{center}
\end{figure}

A fairly big correlation can be seen between popularity and amount of works. Since this data is biased to more modern anime we thought of trying to correlate with more recent works only. But, how recent? Last 5, 10 or 20 years? Thus correlation between popularity and works from different data frames was analyzed, Fig.~\ref{fig:correlationPopRecentWorks}.

\begin{figure}[!hbt]
	\begin{center}
	\includegraphics[width=\columnwidth]{graphics/correlationPopRecentWorks.png}
	\caption{Last \textit{X} years means works from 2018-\textit{X} to present.}
	\label{fig:correlationPopRecentWorks}
	\end{center}
\end{figure}

The best result was given by recent works from last 9 years. Therefore, this definition of recent works was used from there on.

SUBSECTION PARA ESTO DE "TRATAR DE BUSCAR EXPLICACION DETRAS DE 'ULTIMOS 9 ES MEJOR'"
Graphics of some characteristics of works divided by years were made, trying to shed some light over why works from last 9 years were more "important". Fig.~\ref{fig:avgCaracteristicsOfWorks} shows an improvement in average of scores and favorites, with the biggest peak in 2005. Popularity goes down but we need to consider that this metrics are from MAL and it could mean, for example, the parameter is in disuse, instead of representing how people feel about new anime.

\begin{figure}[!hbt]
	\begin{center}
	\includegraphics[width=\columnwidth]{graphics/avgFavorites.png}
	\caption{Average amount of favorites per year.}
	\label{fig:avgFavorites}
	\end{center}
\end{figure}

TODO ADD SCORE, MEMBERS, ETC GRAPHS AND EXPLANATION OF EACH OF THEM

\begin{figure}[!hbt]
	\begin{center}
	\includegraphics[width=\columnwidth]{graphics/worksPerYear_1960-2018.png}
	\caption{Amount of works divided by years which were aired for the first time.}
	\label{fig:amountOfWorksPerYear}
	\end{center}
\end{figure}
As we can see on Fig.~\ref{fig:amountOfWorksPerYear} anime industry is growing bigger each year, of course this is biased by the fact MAL will sure have every adaptation of last year but maybe not for anime from 1980.

The majority of works are from 1990 to 2018 and half of them are distributed over the last 14 years (2014 to 2018) but as far as we can tell there isn’t anything particular over the last 9 years nor on year 2009. Judging by amount of favorites per year it appears that users have been more active in recent years so this could be one of the reasons.

Some interesting enough correlations are shown next
TODO ADD SCATTER PLOTS AND EXPLAIN MORE

\section{Correlation with multiple features}
For this section Scikit-learn, a free software machine learning Python library, was used. The node attributes were divided into categories, leaving four distinct types:

\begin{itemize}
	\item Personal data:
	\begin{itemize}
		\item Debut
		\item Gender
		\item Activity years (2018-debut)
	\end{itemize}
	\item Works data:
	\begin{itemize}
		\item Amount
		\item Top 5 genre
		\item Favorites
		\item Score
		\item Popularity
	\end{itemize}
	\item Recent works data:
	\begin{itemize}
		\item Same as works but for only last 9 years
	\end{itemize}	
	\item Graph data:
	\begin{itemize}
		\item Degree
		\item Betweenness centrality
		\item Closeness
	\end{itemize}
\end{itemize}

Fitting and prediction experiments were run for each category, each combination of 2, 3 and all of them together; using 80\% of seiyuu as train data and the rest as test. This was done for all following models:
\begin{itemize}
	\item DecisionTreeRegressor
	\item DecisionTreeClassifier
	\item LinearRegression
	\item KNeighborsClassifier
	\item LinearDiscriminantAnalysis
	\item GaussianNB
	\item SVM
\end{itemize}

TODO, WRITE THIS AGAIN:
To compare prediction performance mean and median absolute error were used. Unfortunately since popularity variance is really high we observed good results in terms of absolute error but particular predictions were aloof. That's why we ended up using r2\_score for accuracy comparation. 

TODO SHOW TABLE WITH R2 SCORE RESULTS FOR EACH CATEGORY / MODEL AND GROUP OF CATEGORIES

TO ASK: SHOULD I DIVIDE INTO SECTIONS OF GROUPS OF CATEGORIES? yes
\subsection{Only one category}
\begin{table}[!hbt]
	\begin{center}
	\caption{Only one category R2 score results}
	\label{tab:oneCategory}
	\begin{tabular}{|l|c|c|c|c|}
		\hline
		Model / Data & Recent works & Personal & Graph & Work \\ 
		\hline
		DecisionTreeClassifier & 0.11 & -0.04 & 0.06 & -1.3 \\ 
		\hline
		DecisionTreeRegressor & 0.33 & -0.29 & -0.05 & 0.11 \\ 
		\hline
		GaussianNB & 0 & -4.7 & -0.03 & 0 \\ 
		\hline
		KNeighborsClassifier & 0.22 & -0.04 & 0.06 & -0.05 \\ 
		\hline
		LinearDiscriminantAnalysis & 0.47 & -0.04 & 0.32 & -1.04 \\ 
		\hline
		LinearRegression & 0.57 & 0 & 0.28 & 0.42 \\ 
		\hline
		SVM & -0.02 & -0.04 & -0.01 & -0.02 \\ 
		\hline
	\end{tabular}
	\end{center}
\end{table}

\subsection{Groups of two categories}
\subsection{Groups of three categories}
\subsection{All categories}

TODO ADD SOME OF THE GRAPHICS ABOUT FEATURE IMPORTANCE FOR DTC (ONLY BEST ONES) AND EXPLAIN (FOR EACH SUBSECTION)

ADD THE NEW CATEGORIZATION OF SEIYUU AND PREDICTION RESULTS (lo tengo que hacer primero)
