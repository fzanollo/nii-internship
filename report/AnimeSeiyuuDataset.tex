\chapter{Anime/Seiyuu Dataset}

Anime \---like any other animation projects\--- have voice actors to play roles of each character. They usually have multiple seasons or adaptations based on original content \---which can be manga, games, visual novels, etc\--- and it's not uncommon for a character to have always the same actor voicing it. 

Since some time ago anime industry is growing bigger each year and so is seiyuu industry. Seiyuu can become very popular, with a great international fanbase and work in different areas other than voice acting, for example as singers, on theaters, etc.

\section{Wikidata}
Wikidata\footnote{http://wikidata.org/} is a collaboratively edited knowledge base intended to provide a common source of data which can be used by Wikimedia projects such as Wikipedia. The information is stored in RDF format, and can be retrieved in multiple ways, one of them being through a SPARQL endpoint.

Using Wikidata's SPARQL endpoint we retrieved a list of seiyuu. This list contains all persons that have seiyuu as occupation, a total of 6472 entities were obtained\footnote{There's actually 7030 seiyuu in Wikidata but only 6472 of them have an English label (name)}. Gender, birthday and birthplace information was also fetched (last two were not used in the end because it was lacking in the majority of entities).

\section{MyAnimeList}
Wikidata information about seiyuu's works is really incomplete that's why MyAnimeList (MAL)\footnote{https://myanimelist.net/} was used to retrieve voice acting roles and anime information. MAL is a social networking and social cataloging application website with a large database on anime and manga that started in April 6, 2006. Users can make a list of currently watching, watched and/or favorite anime; score, review, comment and recommend similar ones. They can also comment about and favorite people working on the industry (voice actors, directors, editors, etc).

Since only 59 of Wikidata's seiyuu entities had MyAnimeList ID (MALID) property, a matching between Wikidata and MyAnimeList was done using seiyuu's complete name to retrieve the ID for those who was missing. Successfully restoring 3033 MALIDs, giving a total of 3092 seiyuus with that property; 2956 of them having at least one work according to MAL so we are using this subset for our experiments.

Using Jikan API\footnote{https://jikan.docs.apiary.io/\#} and MALID, seiyuu data, voice acting roles and more information about each anime was retrieved. 

An issue to take into account is whether we unify all anime adaptations of the same intellectual property as one or take a single adaptation as a independent work. We chose the later because each adaptation has its own producer, score, popularity, among other information; it would be incorrect to say a seiyuu worked in a popular work when that adaptation didn't have enough fame.\\

\section{Data retrieved}
All in all we were able to retrieve the following information for 2956 seiyuu and 7614 anime.

\begin{itemize}
	\item For Seiyuu:
	\begin{itemize}
		\item Name
		\item Debut (this was obtained from oldest work aired date)
		\item Gender
		\item Popularity (member\_favorites information of MAL)
		\item Work (anime roles with anime information plus wheter is a main role or not)
	\end{itemize}
	\item For Works (Anime):
	\begin{itemize}
		\item Year that began airing
		\item Favorites
		\item Score (from 0 to 10, MAL user based)
		\item Popularity (ranking over all MAL animes)
		\item Members (how many MAL users have it on their list)
		\item Genres
	\end{itemize}
\end{itemize}

It's important to notice that data such as popularity and scores are retrieved from MAL, which is user review based only; it may differ with actual awards winning or professional reviewing of works.

Further, this dataset is biased in favor of more recent anime and seiyuu, since it accounts for more complete data and with better quality. Oldest anime in this dataset is from 1960 having no record about previous ones. Majority of seiyuu's debut are from 1988 which leads us to think information from thereon is more complete. 

The data was stored using Virtuoso server to create a local SPARQL endpoint, mongodb was also used as an intermediate storage (before formatting data as RDF).











