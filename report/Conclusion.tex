As far as we know this is the first time a \textit{social network of seiyuu} and anime is analyzed. We could observe basic characteristics and structure of this social network. \\

Our dataset will be released in RDF format in hopes of improving seiyuu and anime open data; datasets are currently very incomplete or withouth good format. Also, there's not a well known ontology for this type of information; data features are similar to series or movies but is not the same.\\

Since particular predictions were usually wrong (mostly for small values) this couldn't be used in a practical way to predict popularity of new seiyuu nor to recover lost values. Thus popularity prediction and analysis will remain an open question for now since we couldn't get good results with our reduced data and time.\\

Nonetheless we found that \textit{number of main roles} is an important feature, as one can guess by knowing how the industry works.\\

For future works we think it would be good to do:
\begin{itemize}
\item More analysis of Social Network, specially in terms of how it grows.
\item Improve models and data, try new approachs to improve popularity predictions and analysis.
\item Organize seiyuu and anime data using Wikidata ontology and make it open.
\end{itemize}
